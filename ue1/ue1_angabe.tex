% Options for packages loaded elsewhere
\PassOptionsToPackage{unicode}{hyperref}
\PassOptionsToPackage{hyphens}{url}
%
\documentclass[
]{article}
\usepackage{amsmath,amssymb}
\usepackage{iftex}
\ifPDFTeX
  \usepackage[T1]{fontenc}
  \usepackage[utf8]{inputenc}
  \usepackage{textcomp} % provide euro and other symbols
\else % if luatex or xetex
  \usepackage{unicode-math} % this also loads fontspec
  \defaultfontfeatures{Scale=MatchLowercase}
  \defaultfontfeatures[\rmfamily]{Ligatures=TeX,Scale=1}
\fi
\usepackage{lmodern}
\ifPDFTeX\else
  % xetex/luatex font selection
\fi
% Use upquote if available, for straight quotes in verbatim environments
\IfFileExists{upquote.sty}{\usepackage{upquote}}{}
\IfFileExists{microtype.sty}{% use microtype if available
  \usepackage[]{microtype}
  \UseMicrotypeSet[protrusion]{basicmath} % disable protrusion for tt fonts
}{}
\makeatletter
\@ifundefined{KOMAClassName}{% if non-KOMA class
  \IfFileExists{parskip.sty}{%
    \usepackage{parskip}
  }{% else
    \setlength{\parindent}{0pt}
    \setlength{\parskip}{6pt plus 2pt minus 1pt}}
}{% if KOMA class
  \KOMAoptions{parskip=half}}
\makeatother
\usepackage{xcolor}
\usepackage[margin=1in]{geometry}
\usepackage{graphicx}
\makeatletter
\def\maxwidth{\ifdim\Gin@nat@width>\linewidth\linewidth\else\Gin@nat@width\fi}
\def\maxheight{\ifdim\Gin@nat@height>\textheight\textheight\else\Gin@nat@height\fi}
\makeatother
% Scale images if necessary, so that they will not overflow the page
% margins by default, and it is still possible to overwrite the defaults
% using explicit options in \includegraphics[width, height, ...]{}
\setkeys{Gin}{width=\maxwidth,height=\maxheight,keepaspectratio}
% Set default figure placement to htbp
\makeatletter
\def\fps@figure{htbp}
\makeatother
\setlength{\emergencystretch}{3em} % prevent overfull lines
\providecommand{\tightlist}{%
  \setlength{\itemsep}{0pt}\setlength{\parskip}{0pt}}
\setcounter{secnumdepth}{-\maxdimen} % remove section numbering
\ifLuaTeX
  \usepackage{selnolig}  % disable illegal ligatures
\fi
\IfFileExists{bookmark.sty}{\usepackage{bookmark}}{\usepackage{hyperref}}
\IfFileExists{xurl.sty}{\usepackage{xurl}}{} % add URL line breaks if available
\urlstyle{same}
\hypersetup{
  pdftitle={MLE01 - Vertiefende statistische Verfahren},
  pdfauthor={Max Mustermann, Erika Musterfrau, Hans Mustermann},
  hidelinks,
  pdfcreator={LaTeX via pandoc}}

\title{MLE01 - Vertiefende statistische Verfahren}
\usepackage{etoolbox}
\makeatletter
\providecommand{\subtitle}[1]{% add subtitle to \maketitle
  \apptocmd{\@title}{\par {\large #1 \par}}{}{}
}
\makeatother
\subtitle{1. Übungsblatt SS 2024}
\author{Max Mustermann, Erika Musterfrau, Hans Mustermann}
\date{}

\begin{document}
\maketitle

\hypertarget{allgemeine-information}{%
\section{Allgemeine Information}\label{allgemeine-information}}

Alle Aufgaben sind mit R zu lössen. Die Berechnungen sollen
nachvollziehbar und dokumentiert sein. Um die vollständige Punktezahl zu
erreichen, müssen alle Ergebnisse und Fragen entsprechend interpretiert
bzw. beantwortet werden. Code alleine ist nicht ausreichend! Die Abgabe
erfolgt über Moodle entsprechend der Abgaberichtlinien als pdf und Rmd
File. Bitte inkludieren Sie namentlich alle beteiligten
Gruppenmitglieder sowohl im Bericht als auch im Source Code. Die
jeweiligen Datensätze die für diese Übung relevant sind finden Sie
ebenfalls in Moodle.

\hypertarget{lineare-regressionsanalyse-4p}{%
\section{1 Lineare Regressionsanalyse
{[}4P{]}}\label{lineare-regressionsanalyse-4p}}

Für Menschen, die ihren Blutdruck senken wollen, ist eine häufig
empfohlene Vorgehensweise, die Salzaufnahme zu senken. Sie möchten
feststellen, ob es eine lineare Beziehung zwischen Salzaufnahme und
Blutdruck gibt. Sie nehmen 52 Personen in die Stichprobe auf und messen
deren diastolischen Blutdruck (in mmHg) und Natriumausscheidung
(mmol/24h). {[}\href{https://doi.org/10.1136/bmj.297.6644.319}{ref}{]}

\textbf{{[}2P{]} a:} Importieren Sie den Datensatz
\texttt{intersalt.csv}. Erstellen Sie zwei Regressionsmodelle für den
diastolische Blutdruck (bp) in Abhängigkeit der Natriumausscheidung
(na). Das erste Modell soll alle Datenpunkte verwenden. Für das zweite
Modell sollen die vier Datenpunkte mit der geringsten
Natriumausscheidung aus dem Datensatz entfernt werden.

Führen Sie für beide Modelle eine lineare Regressionsanalyse durch, die
folgende Punkte umfasst:

\begin{enumerate}
\def\labelenumi{\roman{enumi})}
\tightlist
\item
  Modellgleichung inklusive 95\% Konfidenzintervall der Modellparameter
\item
  Interpretation des Ergebnisses hinsichtlich Signifikanz und Modellgüte
\item
  Grafische Darstellung der Regressionsgeraden inkl. Konfidenzintervall
\end{enumerate}

Vergleichen Sie beide Modelle. Was können Sie beobachten?

\textbf{{[}2P{]} b:} Lesen Sie den Artikel ``The (Political) Science of
Salt'' und vergleichen Sie damit Ihre Beobachtungen. Gibt es Faktoren
die in Ihren Modellen eventuell nicht berücksichtigt wurden? Wie lautet
die Schlussfolgerung - führt eine Reduktion der Salzaufnahme zu einer
Blutdrucksenkung?

\hypertarget{lineare-regressionsanalyse-kategorisch-3p}{%
\section{2 Lineare Regressionsanalyse (kategorisch)
{[}3P{]}}\label{lineare-regressionsanalyse-kategorisch-3p}}

Der Datensatz \texttt{infant.csv} enthält Information über die
unterschiedliche Kindersterblichkeit zwischen den Kontinenten. Die
Variable \texttt{infant} enthält die Kindersterblichkeit in Tode pro
1000 Geburten. Unterscheidet sich die Kindersterblichkeit zwischen den
Kontinenten?

\textbf{{[}2P{]} a:} Führen Sie eine Regressionsanalyse mit Europa als
Referenz durch, welche die folgenden Punkte umfasst:

\begin{enumerate}
\def\labelenumi{\roman{enumi})}
\tightlist
\item
  Modellgleichung inklusive 95\% Konfidenzintervall der Modellparameter
\item
  Interpretation des Ergebnisses hinsichtlich Signifikanz
\item
  Beurteilung der Modellgüte und Residuenanalyse
\end{enumerate}

\textbf{{[}1P{]} b:} Wie hoch ist die Kindersterblichkeit in Europa und
wie hoch in Afrika (inkl. Unsicherheit)?

\hypertarget{regressionsanalyse-3p}{%
\section{3 Regressionsanalyse {[}3P{]}}\label{regressionsanalyse-3p}}

Die Daten \texttt{wtloss.xlsx} enthalten den Gewichtsverlauf eines
adipösen Patienten im Zuge einer Diät. Sie als betreuender Mediziner und
passionierter Freizeit Data Scientist möchten ein geeignetes
Regressionsmodell erstellen, um den Verlauf der Diät besser steuern zu
können. Das ideale Zielgewicht bezogen auf die Größe des Patienten wäre
bei 80 kg. Importieren Sie den Datensatz mit Hilfe der
\texttt{read\_excel()} Funktion aus dem \texttt{library(readxl)} Paket.

\textbf{{[}2P{]} a:} Die Regressionsanalyse sollte folgende Punkte
inkludieren:

\begin{enumerate}
\def\labelenumi{\roman{enumi})}
\tightlist
\item
  Modellgleichung inklusive 95\% Konfidenzintervall der Modellparameter
\item
  Interpretation des Ergebnisses hinsichtlich Signifikanz
\item
  Beurteilung der Modellgüte und Residuenanalyse
\item
  Grafische Darstellung der Regressionsgeraden inkl. Konfidenz-und
  Vorhersageintervall
\end{enumerate}

\textbf{{[}1P{]} b:} Welches Gewicht hat der Patient nach 30 Tagen bzw.
nach 200 Tagen Diät?

\hypertarget{multiple-regressionsanalyse-3p}{%
\section{4 Multiple Regressionsanalyse
{[}3P{]}}\label{multiple-regressionsanalyse-3p}}

Die Framingham-Herz-Studie war ein Wendepunkt bei der Identifizierung
von Risikofaktoren für koronare Herzkrankheiten und ist eine der
wichtigsten epidemiologischen Studien die je durchgeführt wurden. Ein
großer Teil unseres heutigen Verständnisses von
Herz-Kreislauf-Erkrankungen ist auf diese Studie zurückzuführen. Der
Datensatz \texttt{Framingham.sav} enthält Varibalen hinsichtlich
Demographie, Verhaltensweise, Krankengeschichte und Risikofaktoren.
Finden Sie ein geeignetes Modell, dass den systolischen Blutdruck
(\texttt{sysbp}) beschreibt. Vermeiden Sie nicht relevante bzw.
redundante Variablen (z.B. ``Incident'' Variablen). Achten Sie auf
Ausreißer und fehlende Daten (\texttt{NaN,\ NA\textquotesingle{}s}).

\textbf{{[}2P{]} a:} Die Regressionsanalyse sollte folgende Punkte
inkludieren:

\begin{enumerate}
\def\labelenumi{\roman{enumi})}
\tightlist
\item
  Modellgleichung inklusive 95\% Konfidenzintervall der Modellparameter
\item
  Interpretation des Ergebnisses hinsichtlich Signifikanz
\item
  Beurteilung der Modellgüte und Residuenanalyse
\end{enumerate}

\textbf{{[}1P{]} b:} Welchen systolischen Blutdruck hat eine Person mit
folgendem Profil:

Frau, 50 Jahre, High School, Raucher, 8 Zig/Tag, keine Blutdruck
senkenden Medikamente, 220 mg/dl Serum Cholesterol, 85 mmHg
diastolischer Blutdruck, BMI von 30, kein Diabetes, 90 bpm Herzrate und
Glukoselevel von 90 mg/dl.

\end{document}
